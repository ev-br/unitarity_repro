% DO NOT EDIT - automatically generated from metadata.yaml

\def \codeURL{https://github.com/rescience-c/template}
\def \codeDOI{}
\def \codeSWH{}
\def \dataURL{}
\def \dataDOI{}
\def \editorNAME{}
\def \editorORCID{}
\def \reviewerINAME{}
\def \reviewerIORCID{}
\def \reviewerIINAME{}
\def \reviewerIIORCID{}
\def \dateRECEIVED{01 November 2018}
\def \dateACCEPTED{}
\def \datePUBLISHED{}
\def \articleTITLE{[Re] (Partial) reproduction of PRL 96, 160402 (2006)}
\def \articleTYPE{Replication}
\def \articleDOMAIN{Computational Physics}
\def \articleBIBLIOGRAPHY{bibliography.bib}
\def \articleYEAR{2020}
\def \reviewURL{}
\def \articleABSTRACT{This article reports on a successful exercise in a partial reproduction of a large-scale Monte Carlo simulation of the unitary regime of the Hubbard model. While the main computational code is in Fortran, and is thus largely invariant under evolution of both hardware and software, this exercise highlights the need for human-readable and human-manageable documentation of the computational pipeline.}
\def \replicationCITE{}
\def \replicationBIB{}
\def \replicationURL{}
\def \replicationDOI{}
\def \contactNAME{Evgeni Burovski}
\def \contactEMAIL{evgeny.burovskiy@gmail.com}
\def \articleKEYWORDS{rescience c, Fortran, 10-year-challenge}
\def \journalNAME{ReScience C}
\def \journalVOLUME{4}
\def \journalISSUE{1}
\def \articleNUMBER{}
\def \articleDOI{}
\def \authorsFULL{Evgeni Burovski}
\def \authorsABBRV{E. Burovski}
\def \authorsSHORT{Burovski}
\title{\articleTITLE}
\date{}
\author[1,\orcid{0000-0001-8149-0483}]{Evgeni Burovski}
\affil[1]{National Research University Higher School of Economics, Moscow, Russia}
